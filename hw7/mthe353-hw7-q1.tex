\documentclass[%
  hwnumber=7,%
  studentnumber=20053722,%
  {name=Bryan Hoang}%
]{%
  mthe353answer%
}

\begin{document}
  \begin{questions}
    \setcounter{question}{0}
    \question{}
    \begin{solution}
      To find the distribution of \(X\), let's find the mgf of \(X\).
      \begin{align*}
        M_X(t) &= \E*{e^{tX}}\\
        &= \E*{\sum_{n=0}^\infty \frac{(tX)^n}{n!}}\\
        &= \sum_{n=0}^\infty \frac{\E{X^n}}{n!}t^n
        \intertext{Assuming we can take expectation inside the infinite sum. Then}
        M_X(t) &= 1 + \sum_{n=1}^\infty \frac{\E{X^n}}{n!}t^n\\
        &= 1 + \sum_{n=1}^\infty \frac{\left(\frac{n!}{\lambda^n}\right)}{n!}t^n\\
        &= 1 + \sum_{n=1}^\infty \left(\frac{t}{\lambda}\right)^n\\
        &= \sum_{n=0}^\infty \left(\frac{t}{\lambda}\right)^n\\
        &= \frac{1}{1-\frac{t}{\lambda}} && \text{if}\ \abs{t} < \lambda\\
        &= \left(\frac{\lambda}{\lambda-t}\right)^1 && \text{if}\ \abs{t} < \lambda
      \end{align*}
      which is the mgf of a random variable with a Gamma\((1, \lambda) \sim
      \text{Exponential}(\lambda)\) distribution as seen in lecture. Thus,
      \(\boxed{X \sim \text{Exponential}(\lambda)}\).
    \end{solution}
  \end{questions}
\end{document}
