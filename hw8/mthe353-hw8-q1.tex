\documentclass[%
  hwnumber=8,%
  studentnumber=20053722,%
  {name=Bryan Hoang}%
]{%
  mthe353answer%
}

\begin{document}
  \begin{questions}
    \setcounter{question}{0}
    \question{}
    \begin{parts}
      \part{}\label{part:a}
      \begin{solution}
        \begin{proof}
          Let \(\varepsilon \in \reals_{> 0}\). Then
          \begin{align*}
            \prob{X > M + \varepsilon} &= \prob{X - X_n > M - X_n + \varepsilon}\\
            &= \prob{X - X_n > M - X_n + \varepsilon | M - X_n \ge 0} \prob{M - X_n \ge 0}\\
            &\inviseq  + \prob{X - X_n > M - X_n + \varepsilon | M - X_n < 0} \prob{M - X_n < 0}
          \end{align*}
          by the law of total probability. Then
          \begin{equation*}
            \prob{X > M + \varepsilon} = \prob{X - X_n > M - X_n + \varepsilon | M - X_n \ge 0}
          \end{equation*}
          since \(\prob{\abs{X_n} \le M = 1}\) by assumption. It follows that
          \begin{align*}
            \prob{X > M + \varepsilon} &\le \prob{X - X_n > \varepsilon}\\
            &\le \prob{\abs{X - X_n} > \varepsilon} \to 0
          \end{align*}
          as \(n \to \infty\) since \(X_n \to X\) by assumption. Hence,
          \(\prob{X>M+\varepsilon} = 0\ \forall \varepsilon \in \reals_{>0}\).
          This implies that \(\prob{X>M}=0\), and so \(\prob{X \le M}=1\). A similar
          argument can be used to show that \(\prob{-X < -M - \varepsilon} = 0\), and so
          \(\prob{-X \ge M} = 1\).

          Therefore, \(\prob{\abs{X} \le M} = 1\).
        \end{proof}
      \end{solution}
      \part{}
      \begin{solution}
        \begin{proof}
          Let \(r \in \integers_{>0}\) and let \(B=\{\omega : \abs{X(\omega)} \le M\}\).

          If \(\omega \in A_n(\varepsilon)^c\), then \(\abs{X_n(\omega)-X(\omega)}\le \varepsilon
          \implies \abs{X_n(\omega)-X(\omega)}^r \le \varepsilon^r\).

          If \(\omega \in A_n(\varepsilon) \cap B\), then
          \begin{align*}
            \abs{X_n(\omega)-X(\omega)} &\le \abs{X_n(\omega)} + \abs{X(\omega)}
              && \text{by the triangle inequality}\\
            &\le 2M && \text{since \(\prob{X_n \le M} = 1\) and \(\omega \in B\)}\\
            \implies \abs{X_n(\omega)-X(\omega)}^r &\le (2M)^r
          \end{align*}
          Hence, \(\abs{X_n-X}^r \le \varepsilon^r I_{A_n(\varepsilon)^c} +
          (2M)^r I_{A_n(\varepsilon)}\ \forall \omega \in B\). Since by part~\ref{part:a},
          \(\prob{(B) = 1}\), then
          \begin{equation*}
            \abs{X_n-X}^r \le \varepsilon^r I_{A_n(\varepsilon)^c} + (2M)^r I_{A_n(\varepsilon)}
          \end{equation*}
          with probability 1.

          By taking the expectations of both sides, we get that
          \begin{align*}
            \E{\abs{X_n-X}^r} &\le \E{\varepsilon^r I_{A_n(\varepsilon)^c} + (2M)^r I_{A_n(\varepsilon)}}\\
            &= \varepsilon^r \prob{A_n(\varepsilon)^c} + (2M)^r \prob{A_n(\varepsilon)}\\
            &\le \varepsilon^r + (2M)^r \prob{A_n(\varepsilon)}
          \end{align*}
          \(\forall n \in \integers_{>0}\) because \(\prob{A_n(\varepsilon)^c} \le 1\). Since \(X_n \xrightarrow{p} X\)
          by assumption, it follows that \(\limit{n}{\infty} \prob{A_n(\varepsilon)} = 0\).

          We then have
          \begin{equation*}
            \limit{n}{\infty} \E{\abs{X_n-X}^r} \le \varepsilon^r
          \end{equation*}
          \(\forall \varepsilon \in \reals_{>0}\), which implies that
          \(\limit{n}{\infty} \E{\abs{X_n-X}^r} = 0\).

          Therefore, \(X_n \xrightarrow{r.m.} X\).
        \end{proof}
      \end{solution}
    \end{parts}
  \end{questions}
\end{document}
