\documentclass[hwnumber=4,studentnumber=20053722]{mthe353answer}

\begin{document}
  \begin{questions}
    \setcounter{question}{4}
    \question{}
    For \(n = 0, 1, 2, 3, \dots\), show that
    \begin{equation}
      \label{eq:statement-to-prove}
      \Gamma(n+\frac{1}{2}) = \frac{\sqrt{\pi}(2n)!}{4^{n}n!}
    \end{equation}
    \begin{solution}
      \begin{proof} (Induction)\\
        \emph{Base case}: For \(n = 0\), we have that
        \begin{align*}
          \Gamma(0 + \frac{1}{2}) &= \int_0^\infty y^{\frac{1}{2}-1}
            e^{-y} \dd{y}\\
          &= \int_0^\infty \frac{e^{-y}}{\sqrt{y}} \dd{y}
          \intertext{Let \(u = \sqrt{y}\). We then have}
          \Gamma(\frac{1}{2}) &= 2 \int_0^\infty \underbrace{e^{-u^2}}_I \dd{u}
          \intertext{where \(I\) is related to the pdf of a Normal(0,
            \(\frac{1}{2}\)) random variable. It follows that}
          \Gamma(\frac{1}{2}) &= 2 \left(\frac{1}{2}\right)
            \left(\sqrt{\frac{1}{2}}\sqrt{2\pi}\right)\\
          &= \sqrt{\pi}\\
          &= \frac{\sqrt{\pi}(2 \cdot 0)!}{4^0(0)!}
        \end{align*}
        which means~\eqref{eq:statement-to-prove} is true for \(n = 0\).

        \emph{Inductive step}: Suppose~\eqref{eq:statement-to-prove} holds for
        \(k \in \Z_{\ge 0}\). Then we have that
        \begin{align*}
          \Gamma((k+1)+\frac{1}{2}) &= \int_0^\infty y^{(k+1)+\frac{1}{2}-1}
            e^{-y} \dd{y}\\
          &= \int_0^\infty y^{k+\frac{1}{2}} e^{-y} \dd{y}
          \intertext{Let \(u = y^{k+\frac{1}{2}}\) and \(\dd(v) = e^{-y}\).
            Then integration by parts gives us}
          \Gamma((k+1)+\frac{1}{2}) &=
            \left[\left(k+\frac{1}{2}\right)y^{k-\frac{1}{2}}e^{-y}\right]_0^\infty
            - \int_0^\infty \left(k+\frac{1}{2}\right)(-1)y^{k-\frac{1}{2}} e^{-y} \dd{y}\\
          &= 0 + \left(k+\frac{1}{2}\right) \int_0^\infty y^{k-\frac{1}{2}} e^{-y} \dd{y}\\
          &= \left(k+\frac{1}{2}\right) \Gamma(k+\frac{1}{2})\\
          &= \left(k+\frac{1}{2}\right) \left(\frac{\sqrt{\pi}(2k)!}{4^k k!}\right)
          \intertext{by the inductive hypothesis. Then}
          \Gamma((k+1)+\frac{1}{2}) &= \left(k+\frac{1}{2}\right) \left(\frac{\sqrt{\pi}(2k)!}{4^k k!}\right)
            \left(\frac{4}{4}\right)\left(\frac{k+1}{k+1}\right)
          \intertext{by the twice repeated application of
            \textbf{MULTIPLYING BY 1}! Hence, we have that}
          \Gamma((k+1)+\frac{1}{2}) &= \frac{\sqrt{\pi}(2k)!(2k+2)(2k+1)}
            {4^{k+1}(k+1)!}\\
          &= \frac{\sqrt{\pi}(2k+2)!}{4^{k+1}(k+1)!}\\
          &= \frac{\sqrt{\pi}(2(k+1))!}{4^{k+1}(k+1)!}
        \end{align*}
        Thus,~\eqref{eq:statement-to-prove} holds for \(k+1\).

        Since both the base case and inductive step have been performed, then by
        mathematical induction, the statement~\eqref{eq:statement-to-prove} holds
        for all \(n \in \Z_{\ge 0}\).
      \end{proof}
    \end{solution}
  \end{questions}
\end{document}
