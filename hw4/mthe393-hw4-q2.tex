\documentclass[hwnumber=4,studentnumber=20053722]{mthe353answer}

\begin{document}
  \begin{questions}
    \setcounter{question}{1}
    \question{}
    Finding \(E[X_i X_j], \text{ for } i, j \in \{1, \dots, k\}\).
    \begin{solution}
      Let \(X_i = X_{i,1} + \dots + X_{i,n}\) where
      \begin{align*}
        X_{i,k} &=
        \begin{cases}
          1, & \text{if the \(k\)th multinomial experiment has outcome i}\\
          0, & \text{otherwise}
        \end{cases}
        \intertext{and let \(X_j = X_{j,1} + \dots + X_{j,n}\) where}
        X_{j,k} &=
        \begin{cases}
          1, & \text{if the \(k\)th multinomial experiment has outcome j}\\
          0, & \text{otherwise}
        \end{cases}
      \end{align*}
      We then have
      \begin{align*}
        E[X_i X_j] &= E\left[\left(\sum_{l=1}^n X_{i,l}\right)
          \left(\sum_{m=1}^n X_{j,m}\right)\right]\\
        &= E\left[\sum_{l=1}^n \sum_{m=1}^n X_{i,l} X_{j,m}\right] &&
          \text{by the distributivity of summation}\\
        &= \sum_{l=1}^n \sum_{m=1}^n E\left[X_{i,l} X_{j,m}\right] &&
          \text{by the linearity of expectation}\\
      \end{align*}
      For each \(l,m \in \{1, \dots, n\}\),
      \begin{equation*}
        E\left[X_{i,l}X_{j,m}\right] =
        \begin{cases}
          p_i p_j, & \text{if } l \ne m\\
          0, & \text{if } l = m
        \end{cases}
      \end{equation*}
      Since \(l = m\) only \(n\) times in out of the total \(n^2\) terms in the summation,
      we have that
      \begin{equation*}
        \boxed{E[X_i X_j] = (n^2-n)p_i p_j}
      \end{equation*}
    \end{solution}
  \end{questions}
\end{document}
