\documentclass[%
  hwnumber=5,%
  studentnumber=20053722,%
  {name=Bryan Hoang}%
]{%
  mthe353answer%
}

\begin{document}
  \begin{questions}
    \setcounter{question}{3}
    \question{}
    \begin{solution}
      The first jump from the \(j\)th best vertex has an equal probability of
      \(\frac{1}{j-1}\) to land on any of the other \(j-1\) better vertices
      where those vertices have their expected number of jumps to reach \(B\).
      Then we have the following recursive relationship:
      \begin{align*}
        M_j &= 1 + \frac{1}{j-1}(M_{j-1} + \dotsb + M_2 + M_1)
        \intertext{where \(M_1 = 0\) and \(M_2 = 1\)}
        &= 1 + \frac{1}{j-1} \sum_{i=2}^{j-1} M_i \label{eq:recursive} \numberthis
      \end{align*}
      Let \(P(j)\) be the statement that \(M_j = \sum_{i=1}^{j-1} \frac{1}{i}\) for \(j \in
      \integers_{\ge 2}\). We will prove \(P(j)\) by strong induction.
      \begin{proof} (Strong Induction)\\
        \emph{Base case}: For \(j = 2\), it is clear that
        \begin{align*}
          M_2 = 1 = \frac{1}{1}
        \end{align*}
        Thus, \(P(j)\) is true for \(j = 2\).\\
        \emph{Inductive step}: Suppose \(P(j)\) is true \(\forall j \in \{2, \dotsc, k\}\) for
        some \(k \in \integers_{\ge 2}\). Then
        \begin{align*}
          M_{k+1} &= 1 + \frac{1}{k} \sum_{i=2}^{k} M_i
            && \text{by~\eqref{eq:recursive}}\\
          &= 1 + \frac{1}{k} \sum_{i=2}^{k} \left(\sum_{m=1}^{i-1} \frac{1}{m}\right)
            && \text{by the inductive hypothesis}\\
          &= 1 + \frac{1}{k}\left((k-1)\frac{1}{1} + (k-2)\frac{1}{2} + \dotsb + \frac{1}{k-1}\right)\\
          &= 1 + \frac{1}{k}\left(\sum_{i=1}^{k-1} \frac{k-i}{i}\right)\\
          &= 1 + \frac{1}{k}\left(\left(k\sum_{i=1}^{k-1} \frac{1}{i}\right) - (k-1)\right)\\
          &= 1 + \sum_{i=1}^{k-1} \frac{1}{i} - 1 + \frac{1}{k}\\
          &= \sum_{i=1}^k \frac{1}{i}
        \end{align*}
        Since both the base case and inductive step have been performed, then by
        mathematical induction, the statement \(P(j)\) holds
        for all \(j \in \integers_{\ge 2}\).
      \end{proof}
    \end{solution}
  \end{questions}
\end{document}
