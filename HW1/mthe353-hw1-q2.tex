\documentclass{mthe353answer}

\lhead{
  MTHE/STAT 353 - Homework 1, 2020\\
  \vspace{1em}
  \underline{\makebox[1.35in][c]{20053722}}\\
  Student Number
}

\begin{document}
\begin{questions}
  \setcounter{question}{1}
  \question{}
  \begin{align*}
    f_{X_1,X_2}(x_1,x_2) =& \int_{-\infty}^{\infty} f_X(x_1,x_2,x_3) \,\mathrm{d}x_3\\
    =& \int_{-\infty}^{\infty} \frac{1}{\sqrt{2\pi}} e^{-\frac{(x_1-x_3)^2}{2}} \frac{1}{\sqrt{2\pi}} e^{-\frac{(x_2-x_3)^2}{2}} \frac{1}{\sqrt{2\pi}} e^{-\frac{x_3^2}{2}} \,\mathrm{d}x_3\\
    =& \frac{1}{(2\pi)^\frac{3}{2}} \int_{-\infty}^{\infty} e^{\frac{-x_1^2+2x_1x_3-x_3^2}{2}} e^{\frac{-x_2^2+2x_2x_3-x_3^2}{2}} e^{-\frac{x_3^2}{2}} \,\mathrm{d}x_3\\
    =& \frac{1}{(2\pi)^\frac{3}{2}} \int_{-\infty}^{\infty} e^{\frac{-3x_3^2+2x_1x_3+2x_2x_3-x_1^2-x_2^2}{2}} \,\mathrm{d}x_3 \numberthis \label{eq:1}
  \end{align*}
  Let's try and convert the integrand of the integral in~\eqref{eq:1} into the 
  form \(e^\frac{-(x_3-\mu)^2}{2\sigma^2}\) so that the integrand can be treated 
  as a form of a p.fd.f.\ for a Normal(\(\mu,\sigma^2\)) 
  random variable. Mainly, let's rearrange the integrand's exponent, \textbf{by adding 0}:
  \begin{align*}
     &\; \frac{(-3x_3^2+2x_1x_3+2x_2x_3-x_1^2-x_2^2+\frac{2}{3}x_1^2+\frac{2}{3}x_2^2-\frac{2}{3}x_1x_2+)-\frac{2}{3}x_1^2-\frac{2}{3}x_2^2+\frac{2}{3}x_1x_2}{2}\\
    =&\; \frac{-3\left(x_3^2 - 2x_3(\frac{1}{3}x_1+\frac{1}{3}x_2) + (\frac{1}{9}x_1^2 + 2(\frac{1}{9}x_1x_2) + \frac{1}{9}x_2^2)\right)}{2} - \frac{\frac{2}{3}x_1^2 + \frac{2}{3}x_2^2 - \frac{2}{3}x_1x_2}{2}\\
    =&\; \frac{-\left(x_3-(\frac{1}{3}x_1+\frac{1}{3}x_2)\right)^2}{2(\frac{1}{\sqrt{3}})^2} - \frac{2x_2^2-2x_1x_2+2x_1^2}{2(3)} \numberthis \label{eq:2}
  \end{align*}
  \begin{gather*}
    \text{\eqref{eq:2}} \Rightarrow f_{X_1,X_2}(x_1,x_2) = \frac{1}{(2\pi)^\frac{3}{2}} e^{-\frac{2x_2^2-2x_1x_2+2x_1^2}{2(3)}} 
      \underbrace{\int_{-\infty}^{\infty} e^{\frac{-\left(x_3-(\frac{1}{3}x_1+\frac{1}{3}x_2)\right)^2}{2(\frac{1}{\sqrt{3}})^2}} \,\mathrm{d}x_3}_{I_1}\\
    \Rightarrow \text{The integrand of \(I_1\) is related to the p.d.f.\ of a Normal(\(\frac{1}{3}x_1+\frac{1}{3}x_2, \frac{1}{3}\)) r.v.}\\
    \Rightarrow I_1 = \sqrt{2\pi}\frac{1}{\sqrt{3}}\\
    \Rightarrow 
    \boxed{f_{X_1,X_2}(x_1,x_2) = \frac{1}{\sqrt{2\pi}^2\sqrt{3}}\;e^{-\frac{2x_2^2-2x_1x_2+2x_1^2}{2(3)}},\quad \forall x_1,x_2 \in (-\infty,\infty)}
  \end{gather*}
  Then for the marginal pdf of \(X_1\),
  \begin{align*}
    f_{X_1}(x_1) =& \int_{-\infty}^{\infty} f_{X_1,X_2}(x_1,x_2) \,\mathrm{d}x_2\\
    =& \int_{-\infty}^{\infty} \frac{1}{\sqrt{2\pi}^2\sqrt{3}}\;e^{-\frac{2x_2^2-2x_1x_2+2x_1^2}{2(3)}} \,\mathrm{d}x_2 \\
    =& \frac{1}{\sqrt{2\pi}^2\sqrt{3}} \int_{-\infty}^{\infty} e^{-\frac{2x_2^2-2x_1x_2+2x_1^2}{2(3)}} \,\mathrm{d}x_2 \numberthis \label{eq:3}
  \end{align*}
  Let's do the same thing we did for~\eqref{eq:1} to~\eqref{eq:3}.\ i.e. \textbf{add 0.}
  \begin{align*}
     & \frac{(-2x_2^2+2x_1x_2-2x_1^2+\frac{3}{2}x_1^2)-\frac{3}{2}x_1^2}{2(3)}\\
    =& \frac{-2(x_2^2+2x_2(\frac{1}{2}x_1)+\frac{1}{4}x_1^2)}{2(3)}-\frac{\frac{3}{2}x_1^2}{2(3)}\\
    =& \frac{-(x_2-\frac{1}{2}x_1)^2}{2\sqrt{\frac{3}{2}}^2}-\frac{1}{4}x_1^2 \numberthis \label{eq:4}
  \end{align*}
  \begin{gather*}
    \text{\eqref{eq:4}} \Rightarrow f_{X_1}(x_1) = \frac{1}{\sqrt{2\pi}^2\sqrt{3}}e^{-\frac{1}{4}x_1^2} 
      \underbrace{\int_{-\infty}^{\infty} e^{\frac{-(x_2-\frac{1}{2}x_1)^2}{2\sqrt{\frac{3}{2}}^2}} \,\mathrm{d}x_2}_{I_2}\\
    \Rightarrow \text{The integrand of \(I_2\) is related to the p.d.f.\ of a Normal(\(\frac{1}{2}x_1, \frac{3}{2}\)) r.v.}\\
    \Rightarrow I_2 = \sqrt{2\pi}\sqrt{\frac{3}{2}}\\
    \Rightarrow \boxed{f_{X_1}(x_1) = \frac{1}{\sqrt{2\pi}\sqrt{2}}e^{-\frac{x_1^2}{2(2)}},\quad \forall x_1 \in (-\infty,\infty)}\\
  \end{gather*}
  \hfill\qed{}
\end{questions}
\end{document}
