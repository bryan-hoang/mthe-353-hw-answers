\documentclass[%
  hwnumber=7,%
  studentnumber=20053722,%
  {name=Bryan Hoang}%
]{%
  mthe353answer%
}

\begin{document}
  \begin{questions}
    \setcounter{question}{3}
    \question{}
    \begin{solution}
      \begin{proof}
        Let \(a \in [0, M)\) and \(A = \{X \ge a\}\). Then we claim that
        \begin{equation}
          \label{eq:upper bound}
          X \le M I_A + aI_{A^C}
        \end{equation}
        The inequality in~\eqref{eq:upper bound} is true since if \(I_A = 1\), then the
        inequality becomes
        \begin{equation*}
          X \le M = M(1) + a(0)
        \end{equation*}
        which is true by the hypothesis about \(X\). If \(I_{A^C} = 1\), then
        the inequality becomes
        \begin{equation*}
          X < a = M(0) + a(1)
        \end{equation*}
        which is true when \(I_{A^C} = 1\).

        By taking the expectations of both sides of~\eqref{eq:upper bound}, we
        see that
        \begin{align*}
          \E{X} &\le \E{M I_A + aI_{A^C}}\\
          \E{X} &\le M \E{I_A} + a \E{I_{A^C}}
          \intertext{by the linearity of expectation. Then}
          \E{X} &\le M \prob{X \ge a} + a \prob{X < a}\\
          \E{X} &\le M \prob{X \ge a} + a - a \prob{X \ge a}\\
          \implies \prob{X \ge a}(M - a) & \ge \E{X} - a\\
          \prob{X \ge a} &\ge \frac{\E{X} - a}{M - a} \qedhere
        \end{align*}
      \end{proof}
    \end{solution}
  \end{questions}
\end{document}
