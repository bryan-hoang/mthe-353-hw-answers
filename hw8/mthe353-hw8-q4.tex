\documentclass[%
  hwnumber=8,%
  studentnumber=20053722,%
  {name=Bryan Hoang}%
]{%
  mthe353answer%
}

\begin{document}
  \begin{questions}
    \setcounter{question}{3}
    \question{}
    \begin{solution}
      To convert the product into a sum for the CTL, let's take the natural
      logarithm of both sides of the inequality. A transformation of random
      variables will be necessary, and so let \(h(x) = \ln(x)\) and let \(Y_i = h(X_i)\),
      for \(i \in \{1, \dotsc, 100\}\). Then \(g(y) = h^{-1}(y) = e^y\). With \(f_X(x)\)
      as the common pdf of the \(X_i\)'s, it follows that the common pdf of each
      \(Y_i\) is
      \begin{align*}
        f_Y(y) &= f_X(g(y))\abs{J_g(y)}\\
        &= f_X(e^y) e^y\\
        &= \frac{1}{e} 1_{\{y \le 1\}} e^y\\
        &= e^{y-1} 1_{\{y \le 1\}}
      \end{align*}
      The common expectation, \(\mu\), of each \(Y_i\) is
      \begin{align*}
        \E{Y} &= \int_{-\infty}^{1} y f_Y(y) \dd{y}\\
        &= \int_{-\infty}^{1} y e^{y-1} \dd{y}\\
        &= \left.(y-1)e^{y-1}\right|_{y = -\infty}^{y = 1}\\
        &= 0
      \end{align*}
      and the common second moment of each \(Y_i\) is
      \begin{align*}
        \E{Y^2} &= \int_{-\infty}^{1} y^2 e^{y-1} \dd{y}\\
        &= \left.(y^2-2y+2)e^{y-1}\right|_{y=-\infty}^{y=1}\\
        &= 1
      \end{align*}
      Thus the common variance, \(\sigma^2\), of each \(Y_i\) is
      \begin{align*}
        \Var{Y} &= \E{Y^2} - \E{Y}^2\\
        &= 1
      \end{align*}
      Evaluating the probability yields
      \begin{align*}
        \prob*{\prod_{i=1}^{100} X_i \le c} &= \prob*{\ln(\prod_{i=1}^{100} X_i) \le \ln(c)}\\
        &= \prob*{\sum_{i=1}^{100} Y_i \le \ln(c)}
      \end{align*}
      Let \(S_n = \sum_{i=1}^{100} Y_i\) and \(Z \sim \text{N}(0,1)\). Then by
      the central limit theorem, we have that
      \begin{align*}
        \frac{S_n}{\sqrt{n}} &\xrightarrow{d} Z\\
        \implies \frac{S_{100}}{10} &\dot{\sim} Z
      \end{align*}
      Hence, \(\prob*{\prod_{i=1}^{100} X_i \le c} = \prob{\frac{S_n}{10} \le \frac{1}{10}\ln(c)} = \frac{1}{2}\)
      when \(\frac{1}{10}\ln(c) = 0\) by the symmetry of the normal distribution.

      Therefore, \fbox{\(c = 1\)}.
    \end{solution}
  \end{questions}
\end{document}
