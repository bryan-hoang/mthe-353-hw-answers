\documentclass[%
  hwnumber=8,%
  studentnumber=20053722,%
  {name=Bryan Hoang}%
]{%
  mthe353answer%
}

\begin{document}
  \begin{questions}
    \setcounter{question}{1}
    \question{}\
    \begin{parts}
      \part{}
      \begin{solution}
        \begin{proof}
          Since each \(X_i\) in the sequence has the mean \(\mu = \frac{1}{\lambda}\),
          then by the strong law of large numbers, \(\overline{X}_n \xrightarrow{a.s.}
          \frac{1}{\lambda}\). Let \(f(x) = \frac{1}{x}\). Then \(f(\cdot)\) is
          continuous on \((0,\infty)\), the support of each \(X_i\). Then from
          the lectures, we have that
          \begin{align*}
            f(\overline{X}_n) &\xrightarrow{a.s.} f\left(\frac{1}{\lambda}\right)\\
            \implies Y_n &\xrightarrow{a.s.} \lambda\\
            \implies Y_n &\xrightarrow{p} \lambda \qedhere
          \end{align*}
        \end{proof}
      \end{solution}
      \part{}
      \begin{solution}
        \begin{proof}
          We have that
          \begin{align*}
            \E{{\overline{X}_n-\mu)^2}} &= \Var{\overline{X}_n}\\
            &= \Var{\frac{1}{n}\sum_{i=1}^n X_i}\\
            &= \frac{1}{n^2}\sum_{i=1}^n \Var{X_i}\\
            &= \frac{1}{n^2} \cdot n\sigma^2\\
            &= \frac{\sigma^2}{n}
          \end{align*}
          which implies that \(\limit{n}{\infty} \E{{\overline{X}_n-\mu)^2}} = 0\).

          Therefore, \(\overline{X}_n\) converges to \(\mu\) in mean square.
        \end{proof}
      \end{solution}
    \end{parts}
  \end{questions}
\end{document}
