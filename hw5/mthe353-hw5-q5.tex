\documentclass[%
  hwnumber=5,%
  studentnumber=20053722,%
  {name=Bryan Hoang}%
]{%
  mthe353answer%
}

\begin{document}
  \begin{questions}
    \setcounter{question}{4}
    \question{}\
    \begin{parts}
      \part{}
      \begin{solution}
        \begin{proof}
          Suppose that \(g(\cdot)\) is a function for which
          \(\expect{\expect{Y|X}g(X)}\) and \(\expect{Yg(X)}\) exist. Then
          by LOTUS, we have
          \begin{align*}
            \expect{\expect{Y|X}g(X)} &= \sum_x \expect{Y|X=x} \cdot g(x)
              \cdot p_X(x)\\
            &= \sum_x \sum_y y \cdot p_{Y|X}(y|x) \cdot g(x) \cdot p_X(x)\\
            &= \sum_x \sum_y y \cdot \frac{p_{X,Y}(x,y)}{p_X(x)} \cdot g(x)
              \cdot p_X(x)\\
            &= \sum_x \sum_y y \cdot g(x) \cdot p_{X,Y}(x,y)\\
            &= \expect{Yg(X)}
          \end{align*}
          by LOTUS\@.
        \end{proof}
      \end{solution}
      \part{}
      \begin{solution}
        \begin{proof}
          Suppose that \(\phi(\cdot)\) is a function satisfying
          \(\expect{\phi(X)g(X)}=\expect{Yg(X)}\) for all functions \(g(\cdot)\)
          for which the expectations exist. Then WLOG, consider the function
          \begin{equation*}
            g_x(X) = \begin{cases}
              1, & \text{if}\ X=x\\
              0, & \text{otherwise}
            \end{cases}
          \end{equation*}
          Then
          \begin{align*}
            \expect{\phi(X)g_x(X)} &= \phi(x)
            \intertext{and}
            \expect{Yg_x(X)} &= \expect{Y|X=x}
          \end{align*}
          Therefore, we have that
          \begin{align*}
            \phi(x) &= \expect{Y|X=x}\\
            \implies \phi(X) &= \expect{Y|X}
          \end{align*}
          with probability 1.
        \end{proof}
      \end{solution}
    \end{parts}
  \end{questions}
\end{document}
