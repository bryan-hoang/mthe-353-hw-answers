\documentclass{mthe353answer}

\lhead{
  MTHE/STAT 353 - Homework 2, 2020\\
  \vspace{1em}
  \underline{\makebox[1.35in][c]{20053722}}\\
  Student Number
}

\begin{document}
  \begin{questions}
    \setcounter{question}{2}
    \question{}\
    \begin{subparts}

      \subpart{}
      Finding the joint pmf of \((X_1, X_2, X_3, X_4),\; p(x_1, x_2, x_3, x_4)\).

      \begin{note}
        The majority of the number crunching discussed here will be done in
        part~\ref{subpart:pmf} below.
      \end{note}
      Now, the first thing to observe is that the total number of possible
      arrangements for the placement of the balls in the boxes is \(4! = 24\).
      We can then think about the situation as a combinatorial problem, where
      the probability that a given configuration of fixed points occur is equal
      to the number of permutations which satisfy the fixed point condition
      divided by the total number of permutations possible. The notion of a
      \textbf{derangement} appears during some of the calculations in
      part~\ref{subpart:pmf}.

      The support of the joint pmf can be characterized as
      \begin{equation*}
        \supp{} (p) = \{(x_1, x_2, x_3, x_4) \in \mathbb{Z}^n \mid x_i = 0 \text{ or } 1,\; i \in {1, 2, 3, 4} \text{ and } x_1 + x_2 + x_3 + x_4 \neq 3\}
      \end{equation*}
      which contains 12 points.

      The calculations for the joint pmf at \(k = 0\) and \(k = 4\) are seen in part~\ref{subpart:pmf},
      while the values for that pmf at \(k = 1\) and \(k = 2\) were obtained by
      dividing the pmf of \(X\), \(p_X\), at 1 and 2 by the corresponding number of points
      in the support of \(p\). It then follows that the joint pmf is given by
      \begin{equation*}
        \boxed{
          p(x_1, x_2, x_3, x_4) = \left\{
          \begin{aligned}
            &\frac{9}{24}, & \text{if } x_i = 0 \text{ or } 1,\; i \in {1, 2, 3, 4},\; x_1 + x_2 + x_3 + x_4 = 0 && \text{(1 point)}\\
            &\frac{1}{12}, & \text{if } x_i = 0 \text{ or } 1,\; i \in {1, 2, 3, 4},\; x_1 + x_2 + x_3 + x_4 = 1 && \text{(4 points)}\\
            &\frac{1}{24}, & \text{if } x_i = 0 \text{ or } 1,\; i \in {1, 2, 3, 4},\; x_1 + x_2 + x_3 + x_4 = 2 && \text{(6 points)}\\
            &\frac{1}{24}, & \text{if } x_i = 0 \text{ or } 1,\; i \in {1, 2, 3, 4},\; x_1 + x_2 + x_3 + x_4 = 4 && \text{(1 point)}\\
            &0,            & \text{otherwise}                     && \text{(12 points total)}
          \end{aligned}
          \right.
        }
      \end{equation*}

      \subpart{}
      \label{subpart:pmf}
      Finding the pmf of \(X,\; p_X(x)\), where \(X\) can be interpreted as the
      number of balls placed into their corresponding box.

      The support of \(p_X\) is the set \(\{0, 1, 2, 3, 4\}\). Let's calculate
      \(p_X(x)\) for \(x \in \{0, 1, 2, 4\}\). 3 is not a part of the support
      since it it is impossible to have 3 of the balls belong to their
      corresponding box and for the fourth one to not belong to its box.
      \begin{align*}
        \shortintertext{For \(x = 0\),}
        p_X(0) =&\; \frac{\binom{4}{0}!4}{4!} && \because\ \text{it is the
          derangement of 4 objects}\\
        =&\; \frac{\lfloor \frac{24}{e} + \frac{1}{2} \rfloor}{24}\\
        =&\; \frac{9}{24}
        \shortintertext{For \(x = 1\),}
        p_X(1) =&\; \frac{\binom{4}{1}!3}{24} && \because\ \text{one ball is
          fixed, while the other 3 are om a derangement}\\
        =&\; \frac{4\lfloor \frac{3!}{e} + \frac{1}{2} \rfloor}{24}\\
        =&\; \frac{(4)(2)}{24}\\
        =&\; \frac{1}{3}
        \shortintertext{For \(x = 2\)}
        p_X(2) =&\; \frac{\binom{4}{2}!2}{24}\\
        =&\; \frac{(6)\lfloor \frac{2!}{e} + \frac{1}{2} \rfloor}{24}\\
        =&\; \frac{(6)(1)}{24}\\
        =&\; \frac{1}{4}
        \shortintertext{For \(x = 4\),}
        p_X(4) =&\; \frac{\binom{4}{4}}{24}\\
        =&\; \frac{1}{24}
      \end{align*}
      Therefore,
      \begin{equation*}
        \boxed{
          p_X(x) =
          \begin{cases}
            \frac{9}{24}, & \text{if } x = 0\\
            \frac{1}{3},  & \text{if } x = 1\\
            \frac{1}{4},  & \text{if } x = 2\\
            \frac{1}{24}, & \text{if } x = 4\\
            0, & \text{otherwise}
          \end{cases}
        }
      \end{equation*}
      \subpart{}
      Finding \(E[X]\).
      \begin{align*}
        E[X] =&\; \sum_{x \in \{0, 1, 2, 4\}} xp_X(x)\\
        =&\; (0)(\frac{9}{24}) + (1)(\frac{1}{3}) + (2)(\frac{1}{4}) + (4)(\frac{1}{24})\\
        =&\; \frac{1}{3} + \frac{1}{2} + \frac{1}{6}\\
        \alignedbox{=}{1}
      \end{align*}
    \end{subparts}
  \end{questions}
\end{document}
